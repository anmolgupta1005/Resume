%%%%%%%%%%%%%%%%%%%%%%%%%%%%%%%%%%%%%%%
% Deedy - One Page Two Column Resume
% LaTeX Template
% Version 1.1 (30/4/2014)
%
% Original author:
% Debarghya Das (http://debarghyadas.com)
%
% Original repository:
% https://github.com/deedydas/Deedy-Resume
%
% IMPORTANT: THIS TEMPLATE NEEDS TO BE COMPILED WITH XeLaTeX or LuaLaTeX
%
% This template uses several fonts not included with Windows/Linux by
% default. If you get compilation errors saying a font is missing, find the line
% on which the font is used and either change it to a font included with your
% operating system or comment the line out to use the default font.
% 
%%%%%%%%%%%%%%%%%%%%%%%%%%%%%%%%%%%%%%
% 
% TODO:
% 1. Integrate biber/bibtex for article citation under publications.
% 2. Figure out a smoother way for the document to flow onto the next page.
% 3. Add styling information for a "Projects/Hacks" section.
% 4. Add location/address information
% 5. Merge OpenFont and MacFonts as a single sty with options.
% 
%%%%%%%%%%%%%%%%%%%%%%%%%%%%%%%%%%%%%%
%
% CHANGELOG:
% v1.1:
% 1. Fixed several compilation bugs with \renewcommand
% 2. Got Open-source fonts (Windows/Linux support)
% 3. Added Last Updated
% 4. Move Title styling into .sty
% 5. Commented .sty file.
%
%%%%%%%%%%%%%%%%%%%%%%%%%%%%%%%%%%%%%%%
%
% Known Issues:
% 1. Overflows onto second page if any column's contents are more than the
% vertical limit
% 2. Hacky space on the first bullet point on the second column.
%
%%%%%%%%%%%%%%%%%%%%%%%%%%%%%%%%%%%%%%

\documentclass[]{deedy_format_Anmol}


\begin{document}

%%%%%%%%%%%%%%%%%%%%%%%%%%%%%%%%%%%%%%
%
%     LAST UPDATED DATE
%
%%%%%%%%%%%%%%%%%%%%%%%%%%%%%%%%%%%%%%
\lastupdated

%%%%%%%%%%%%%%%%%%%%%%%%%%%%%%%%%%%%%%
%
%     TITLE NAME
%
%%%%%%%%%%%%%%%%%%%%%%%%%%%%%%%%%%%%%%


\namesection{Anmol}{Gupta}
{\href{mailto:agupta10@bu.edu}{agupta10@bu.edu} | 617 712 8575 | 
LinkedIn:  \href{https://www.linkedin.com/in/anmol-gupta-90b67a108}{\custombold{Anmol Gupta}} |
GitHub:   \href{https://github.com/anmolgupta1005/}{\custombold{anmolgupta1005}}
}

%%%%%%%%%%%%%%%%%%%%%%%%%%%%%%%%%%%%%%
%
%     COLUMN ONE
%
%%%%%%%%%%%%%%%%%%%%%%%%%%%%%%%%%%%%%%

\begin{minipage}[t]{0.325\textwidth} 

%%%%%%%%%%%%%%%%%%%%%%%%%%%%%%%%%%%%%%
%     EDUCATION
%%%%%%%%%%%%%%%%%%%%%%%%%%%%%%%%%%%%%%

\section{Education} 
\vspace{0.5mm} % Hacky fix for awkward extra vertical space 
\runsubsection{Boston University}\\
\location{Boston, MA}
\descript{MS with Thesis \\ Computer Engineering}
Graduation May 2017 \\ GPA: 3.73
\sectionsep

\runsubsection{Mumbai University}\\		
\location{Mumbai, India}
\descript{BE \\ Electronics Engineering}
Graduated - May 2015 \\ GPA: 3.72
\sectionsep

\sectionsep
%%%%%%%%%%%%%%%%%%%%%%%%%%%%%%%%%%%%%%
%     COURSEWORK
%%%%%%%%%%%%%%%%%%%%%%%%%%%%%%%%%%%%%%

\section{Coursework}
\vspace{0.5mm} % Hacky fix for awkward extra vertical space 
\flushleft
\textbullet \, Computer Architecture\\
\textbullet \, Digital VLSI \\
\textbullet \, Verilog and FPGA\\
\textbullet \, Multi-core CPUs \& GPUs\\
\textbullet \, Embedded Systems\\
\textbullet \, Operating Systems\\
\textbullet \, Cybersecurity\\
\textbullet \, Machine Learning
\sectionsep


%%%%%%%%%%%%%%%%%%%%%%%%%%%%%%%%%%%%%%
%     Architectures
%%%%%%%%%%%%%%%%%%%%%%%%%%%%%%%%%%%%%%
\section{Hardware}
\vspace{0.5mm} % Hacky fix for awkward extra vertical space 
\flushleft
\textbullet \, RISCV\\
\textbullet \, Intel x86\\
\textbullet \, ARM\\
\textbullet \, MIPS\\

\sectionsep

%%%%%%%%%%%%%%%%%%%%%%%%%%%%%%%%%%%%%%
%     Programming Languages
%%%%%%%%%%%%%%%%%%%%%%%%%%%%%%%%%%%%%%

\section{Software}
\begin{tabular}{lll}
\custombold{Proficient} & \custombold{Mid} & \custombold{Familiar} \\
C/Cpp& Matlab & MySQL\\
Assembly & Cuda & Perl\\
Verilog & OpenMP & HTML\\
Python & OpenCL \\
Bash &JAVA  \\
SystemVerilog &\LaTeX \\ 
\end{tabular}
\sectionsep

%%%%%%%%%%%%%%%%%%%%%%%%%%%%%%%%%%%%%%
%     TOOLS
%%%%%%%%%%%%%%%%%%%%%%%%%%%%%%%%%%%%%%

\section{Tools} 
\vspace{0.5mm} % Hacky fix for awkward extra vertical space 
\begin{tabular}{ll}
Cadence Virtuoso &Git\\
Scikit-learn &PyMTL\\
Xilinx ISE/Vivado &Overleaf
\end{tabular}
\sectionsep


%%%%%%%%%%%%%%%%%%%%%%%%%%%%%%%%%%%%%%
%     TA
%%%%%%%%%%%%%%%%%%%%%%%%%%%%%%%%%%%%%%
\section{Teaching Assistant}
\vspace{0.5mm} % Hacky fix for awkward extra vertical space 
\textbullet \, Digital VLSI Circuit Design\\
\textbullet \, Introduction to Electronics\\
\textbullet \, Computer Architecture
\sectionsep


\end{minipage} 
\hfill
\begin{minipage}[t]{0.66\textwidth} 
%%%%%%%%%%%%%%%%%%%%%%%%%%%%%%%%%%%%%%
%
%     COLUMN TWO
%
%%%%%%%%%%%%%%%%%%%%%%%%%%%%%%%%%%%%%%


%%%%%%%%%%%%%%%%%%%%%%%%%%%%%%%%%%%%%%
%     EXPERIENCE
%%%%%%%%%%%%%%%%%%%%%%%%%%%%%%%%%%%%%%

\section{Experience}
\vspace{0.5mm} % Hacky fix for awkward extra vertical space 
\runsubsection{\href{https://www.bu.edu/icsg/}{Integrated Circuits \& Systems Group}}
\descript{| Research Assistant }
\location{Boston University, Boston, MA | May-16 to Present}
\vspace{\topsep} % Hacky fix for awkward extra vertical space
\vspace{1mm}
\justify
\begin{tightemize}
\item Working with \textbf{\href{https://www.bu.edu/eng/profile/ajay-joshi/}{Prof. Ajay Joshi}} and \textbf{\href{https://www.bu.edu/eng/profile/manuel-egele/}{Prof Manuel Egele}} on design and development of hardware-based Security
\end{tightemize}
\vspace{\topsep}
\sectionsep

\runsubsection{\href{https://www.siemens.com/in/en/home.html}{Siemens, Ltd.}}
\descript{| PLC Design Intern }
\location{Mumbai, India | June-13 to July-13}
\vspace{1mm}
\justify
\begin{tightemize}
\item Design and Programming of injection modules on S7200 and S7300 PLCs at the Contactors and Relays Manufacturing Unit
%\item Control Panel design of the Injection and Water Pumping Module
\end{tightemize}
\vspace{\topsep}
\sectionsep

%%%%%%%%%%%%%%%%%%%%%%%%%%%%%%%%%%%%%%
%     RESEARCH
%%%%%%%%%%%%%%%%%%%%%%%%%%%%%%%%%%%%%%

\section{Research Project}
\vspace{0.5mm} % Hacky fix for awkward extra vertical space 
\runsubsection{Malware Detection using HPC\smallercaps{s}}
\location{| May-16 – Present}
\vspace{1mm}
\justify
\begin{tightemize}
\item Examine the use of Hardware Performance Counters (HPCs) with supervised machine learning techniques for malware detection 
\item Goal is to prove that it is not possible to classify high-level behavior of a program (whether it is malware or not) using the profiles from HPCs
\end{tightemize}
\vspace{\topsep}
\sectionsep


%%%%%%%%%%%%%%%%%%%%%%%%%%%%%%%%%%%%%%
%     PROJECTS
%%%%%%%%%%%%%%%%%%%%%%%%%%%%%%%%%%%%%%

\section{Projects}
\vspace{0.5mm} % Hacky fix for awkward extra vertical space 

\runsubsection{Security Assessment of Bitcoin}
\location{| December 2016}
\vspace{1mm}
\justify
\begin{tightemize}
\item As a security assessment for the cybersecurity course, highlighted the various attack surfaces that are exploited by Wallet Vulnerabilities, Time Jacking and Transaction Malleability
\item Demonstrated the exploit used in the famous \href{http://www.darkreading.com/attacks-and-breaches/mt-gox-bitcoin-meltdown-what-went-wrong/d/d-id/1114091}{Mt. Gox} attack 
\end{tightemize}
\vspace{\topsep}
\sectionsep

\runsubsection{MBTA Live Tracker}
\location{| April 2016}
\vspace{1mm}
\justify
\begin{tightemize}
\item Designed an embedded system with touch screen based GUI for live tracking of Boston’s public transport system - the MBTA, using GUMSTIX and RASPBERRY PI controllers. \href{https://www.youtube.com/watch?v=DtY4qqCeVRI}{Video Demo Link}
\end{tightemize}
\vspace{\topsep}
\sectionsep

\runsubsection{Mini OS on bare-metal }
\location{| April 2016}
\vspace{1mm}
\justify
\begin{tightemize}
\item Implemented a basic operating system (OS) on bare-metal
\item The OS booted from a GNU-GRUB2 Multiboot Loader, loaded a file-system on the RAM and used a FIFO scheduler to schedule processes
\end{tightemize}
\vspace{\topsep}
\sectionsep

\runsubsection{Rush Hour }
\location{| December 2015}
\vspace{1mm}
\justify
\begin{tightemize}
\item Interfaced a keyboard and a HDMI display monitor to Nexys-3 (based on Xilinx Spartan-6 LX16 FPGA) board to make a video game.  \href{https://youtu.be/s36H25OkzVQ}{Game play in the Link}
\end{tightemize}
\vspace{\topsep}
\sectionsep

\runsubsection{4-AXIS SCARA BOT  }
\location{| May 2015}
\descript{Senior Design Project}
\vspace{1mm}
\justify
\begin{tightemize}
\item A SCARA bot is rigid in all but Z direction. 
\item It had a complex mechanical design with load capabilities up to 10 kg (22 lbs) and the motors were programmed to rotate at 0.1 radians precision
\end{tightemize} 
\vspace{\topsep}
\sectionsep

%%%%%%%%%%%%%%%%%%%%%%%%%%%%%%%%%%%%%%
%     AWARDS AND EXTRA-CURRICULARS
%%%%%%%%%%%%%%%%%%%%%%%%%%%%%%%%%%%%%%

\section{Awards and Extra-Curricular}
\vspace{\topsep} % Hacky fix for awkward extra vertical space
\vspace{1mm}
\justify
\begin{tabular}{ll}
2014 & 1\textsuperscript{st} Position in Prakalpa-14 (national level project exhibition) \\
& for the project `Securing Home Automation using Dropbox'\\
2012-2015 & Technical Head and Member of IEEE-KJSCE Student Chapter \\
\end{tabular}

\end{minipage} 
\end{document}  \documentclass[]{article}
